\documentclass[letterpaper,12pt]{article}
\usepackage[margin=64pt]{geometry}
\usepackage{amsthm}
\usepackage{amsmath}
\usepackage{amssymb}
\usepackage{parskip}
\usepackage{graphicx}
\usepackage{enumerate}
\usepackage{hyperref}
\usepackage{fancyhdr}
\newcommand{\transpose}{^{\mbox{\tiny T}}}


\begin{document}
\thispagestyle{fancy}
\renewcommand{\headrulewidth}{0pt}
\rhead{Chris Saechao}
\hrule \vspace{0.5em}
\noindent {\bf CFRM 462: Introduction to Computational Finance and Financial Econometrics} \hfill Homework 1 \newline \hrule

\begin{enumerate}
\item $\frac{27-31.18}{31.18} = -.134	$
\subitem{a)} If you invested \$10,000 then would lose 10,000 * -.134 and your investment would be worth 10,000 * (1-.134) = \$ 8659.39

\item ln(27) - ln(31.18) = -.144
\subitem{a)} $e^{-.144}$ - 1 = -.134

\item $(1-.134)^{12} - 1$ = - 0.822
\item -.144 * 12 = -1.727

\item $R_t(12) = \frac{30.51 - 31.18}{31.18} = -0.0214$
\subitem{a)} The investment would be worth 10,000 *(1-.0214) = 9785.11. Compared to 3) the nominal amount is higher. 

\item ln(30.51) - ln(31.18) = -.0217
The cc return amount is higher than 4) because the ending balance is higher and it does not assume a monthly decline of 13.4\%
\subitem{a)} $e^{-.0217} - 1 = -0.0214$
\end{enumerate}
PART II \& III refer to supporting files

\end{document}